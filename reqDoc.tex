\documentclass[a4paper,11pt,notitlepage]{article}
\usepackage{xspace}
\usepackage[margin=1in]{geometry}
\usepackage{amsmath}
\usepackage{graphicx}
\usepackage{paralist}
\usepackage{hanging}
\usepackage{enumerate}
\usepackage{color}
\def\TSL{Train Scripting Language\xspace}
\def\CS{Control Progam\xspace}
\def\LC{Local Copy of Train State\xspace}
\title{{\bf Software Project Requirements Specification} \\ for \\ {\bf Train Scripting Python Library}}
\author{Prepared by Alex Wipf, Sebastian Imlay, Max Aussendorf, Paul Sheraton}
\parskip 7.2pt

\begin{document}
\maketitle
\pagebreak
\tableofcontents

%\addtocounter{page}{-1}
\section{Scope}
\subsection{Purpose}
Creation of a Python library to communicate with the control software for the model train in CF408. This library will allow users to create scripts which can control the trains without human interaction.
\subsection{Document Definitions}
\paragraph{\TSL} This refers to the Python with our library added on.
\paragraph{\CS} This refers to the AdaRail program which works between the virtual throttles (such as our Scripting Language) and the track itself. This is the program our library will directly communicate with.
\paragraph{\LC} Our library shall have a thread which listens to all messages, and update a local copy of the train states. For accessor functions, this local copy can be the one read.
\subsection{References}

\section{Vision}
Paul
\subsection{Background}
\subsection{Product Perspective}
\subsection{Risk Assessment}

\section{General Description}
\subsection{User Characteristics} 
Users of this library will be programmers, and will be interested in the model trains. Since this is built on top of Python, anyone using it should be expected to know a bit of python before they start using our library. While the learning curve may be steeper than that of a manual throttle, this library will be much more powerful. We have specified objects and functionality in such a way that it should be intuative to manipulate the train system.
\subsection{Library Features}
Upon inclusion of the library, a user will be able to either start the \CS, or receive from it information about the current state of the train system. An XML document with the physical topology will have to be read in. The user will be able to call library functions, declare objects, and call methods on those objects. All of these features will be to allow the user to poll and manipulate the train system.

\section{Specific Requirements}
Alex
\subsection{Objects in Library}
\subsection{Functions in Library}
\subsubsection{Event Driven Functions}
Object: TrainEvent
\\t = TrainEvent(f1, f2, n)
\\f1, f2 are functions. f1 returns a boolean, f2 does the event's desired things.
\\n is the number of times the event can happen before it is automatically unregistered (0 == infinite)
\\functions on TrainEvents:
\\t.register()		--adds to observer's list of things to watch (no op if already exists)
\\t.unregister()	--removes from observer's list (no op if not exists)

\section{External Interface Requirements}
Sebastian
\subsection{Control Program}
\subsection{Train Topology}

\section{Additions to Current Software}
Sebastian
\subsection{Design Constraints}
\subsection{Control Program}
\subsection{XML Specification}

\section{Non-Functional Requirements}
\subsection{Performance Requirements}
\subsection{Security Requirements}
\subsection{Safety Requirements}

\section{Additional Functionality For Future Releases}
\subsection{Front End}
\subsection{Speed Limits}

\section{Sample Scripts}
Paul
\subsection{Initialization}
\subsection{Starting While Running}
\subsection{Slow Down Over Bridges}
\subsection{Go to Mountains}
\subsection{Manual Switch Control}

\end{document}
