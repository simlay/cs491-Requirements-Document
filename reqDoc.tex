\documentclass[a4paper,11pt,notitlepage]{article}
\usepackage{xspace}
\usepackage[margin=1in]{geometry}
\usepackage{amsmath}
\usepackage{graphicx}
\usepackage{paralist}
\usepackage{hanging}
\usepackage{enumerate}
\usepackage{color}
\def\TSL{Train Scripting Language\xspace}
\def\CS{Control Progam\xspace}
\def\LC{Local Copy of Train State\xspace}
\title{{\bf Software Project Requirements Specification} \\ for \\ {\bf Train Scripting Python Library}}
\author{Prepared by Alex Wipf, Sebastian Imlay, Max Aussendorf, Paul Sheraton}
\parskip 7.2pt

\begin{document}
\maketitle
\pagebreak
\tableofcontents

%\addtocounter{page}{-1}
\section{Scope}
\subsection{Purpose}
Creation of a Python library to communicate with the control software for the model train in CF408. This library will allow users to create scripts which can control the trains without human interaction.
\subsection{Document Definitions}
\paragraph{\TSL} This refers to the Python with our library added on.
\paragraph{\CS} This refers to the AdaRail program which works between the virtual throttles (such as our Scripting Language) and the track itself. This is the program our library will directly communicate with.
\paragraph{\LC} Local copy of all the things.
\subsection{References}

\section{Vision}
Paul
\subsection{Background}
\subsection{Product Perspective}
\subsection{Risk Assessment}

\section{General Description}
Alex
\subsection{Library Features}
\subsection{User Characteristics}

\section{Specific Requirements}
Alex
\subsection{Objects in Library}
\subsection{Functions in Library}
\subsubsection{Event Driven Functions}
Object: TrainEvent
\\t = TrainEvent(f1, f2, n)
\\f1, f2 are functions. f1 returns a boolean, f2 does the event's desired things.
\\n is the number of times the event can happen before it is automatically unregistered (0 == infinite)
\\functions on TrainEvents:
\\t.register()	--adds to observer's list of things to watch (no op if already exists)
\\t.unregister()	--removes from observer's list (no op if not exists)

\section{External Interface Requirements}
Sebastian
\subsection{Control Program}
\subsection{Train Topology}

\section{Additions to Current Software}
Sebastian
\subsection{Design Constraints}
\subsection{Control Program}
\subsection{XML Specification}

\section{Non-Functional Requirements}
\subsection{Performance Requirements}
\subsection{Security Requirements}
\subsection{Safety Requirements}

\section{Additional Functionality}
\subsection{Front End}
\subsection{Speed Limits}

\section{Sample Scripts}

\subsection{Initialization}
\begin{verbatim}
/*
 * Trains must be stopped, and we'll send signals to turn lights/horn off,
 * regardless of state.
 */

if isControlProgramRunning():
    error(“already running”);
runController
readXML(filename)
vector<train> trains = new vector<train>
t1 = new Train(244, 3, 4, 5, 6)    //initialized with ID and vector of sensors owned by it
trains.add(t1)
t2 = new Train(133, 19, 20, 24, 25, 26)
trains.add(t2)
t3 = new Train(212, 10, 11, 12)
trains.add(t3)
initControl(trains)     //sends msg to controller to start up, w/initial train locs
\end{verbatim}

\subsection{Starting While Running}
\begin{verbatim}
/*
 * Trains are assumed to be moving, and here we shall set up the initial 
 * train objects based on querying the control program.
 */

readXML(filename)
if !isControlProgramRunning():
    error(“ctrl not running”)
	trains = getTrainState();
\end{verbatim}

\subsection{Slow Down Over Bridges}
\begin{verbatim}
/*
 * We've decided that we want to slow all trains down while going over
 * bridges for safety.
 */

//
train1 = track.getTrain(1204)
train1.setSpeed(50)

# Get the sections on the bridge(s)
bridgeSections = track.getSections([(15,25), (23,44),(43,66)])

# Loop over all bridge sections and slow the train at all bridge sections
for section in bridgeSections:

when train1 enters section do:
        s = train1.getSpeed()
		if (s > 25):
			train1.setSpeed(25)

			waitUntil(train1.location != section)
			    train1.setSpeed(s)
\end{verbatim}
\subsection{Go to Mountains}
\begin{verbatim}
\end{verbatim}
\subsection{Manual Switch Control}
\begin{verbatim}
\end{verbatim}

\end{document}
